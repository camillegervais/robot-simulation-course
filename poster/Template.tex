\documentclass[a0,landscape]{a0poster}
\usepackage[utf8]{inputenc}
\usepackage{graphicx}
\usepackage{multicol}
\usepackage{amsmath}
\usepackage[table]{xcolor}
\usepackage{enumitem}
\usepackage{tikz}
\usepackage{booktabs}
\usepackage{mdframed}

% Définition du style militaire pour les boîtes
\newmdenv[
    backgroundcolor=militarygreen, % Fond vert militaire pour les boîtes
    linecolor=white, % Bordure blanche
    fontcolor=white, % Texte en blanc
    linewidth=2pt,
    roundcorner=10pt, % Angles arrondis
    topline=false,
    bottomline=false,
    leftline=true,
    rightline=true,
    skipabove=10pt,
    skipbelow=10pt,
    innertopmargin=10pt,
    innerbottommargin=10pt,
    innerleftmargin=10pt,
    innerrightmargin=10pt
]{militarybox}

\usepackage{fontawesome}
\usepackage{geometry}
\usepackage{anyfontsize} % Pour permettre n'importe quelle taille de police

% Réduire les marges pour tout faire tenir sur une page
\geometry{margin=1cm}

% Couleurs militaires
\definecolor{militarygreen}{RGB}{53, 94, 59} % Vert militaire
\definecolor{khaki}{RGB}{189, 183, 107} % Kaki
\definecolor{darkgray}{RGB}{45, 45, 45} % Gris foncé
\definecolor{deserttan}{RGB}{210, 180, 140} % Beige désert
\definecolor{camouflage}{RGB}{120, 134, 107} % Camouflage
\definecolor{militaryblue}{RGB}{55, 94, 151} % Bleu militaire
\definecolor{burgundy}{RGB}{128, 0, 32} % Bordeaux (rouge militaire)

% Ajout d'un fond crème pour la page
\pagecolor{deserttan}

% Titres avec taille réduite
\renewcommand{\section}[1]{\noindent\color{militarygreen}\Large\bfseries #1\\[0.5ex]}
\renewcommand{\subsection}[1]{\noindent\color{militarygreen}\large\bfseries #1\\[0.3ex]}

% Marges
\setlength{\columnsep}{1.5cm}
\setlength{\columnseprule}{0pt}

% Réduire l'espacement des éléments de liste
\setlist{noitemsep,topsep=0pt,parsep=0pt,partopsep=0pt,leftmargin=*}

\begin{document}
\begin{militarybox}
\begin{center}
    {\LARGE\bfseries\sffamily \textcolor{khaki}{SIMULATION DE ROBOTS MILITAIRES}}\\[0.5ex]
    \normalsize\textbf{\textcolor{white}{Groupe n°\ldots:}} \textcolor{deserttan}{Étudiant 1, Étudiant 2, Étudiant 3, Étudiant 4}\\[1ex]
    
    % Logo CentraleSupélec
    \includegraphics[height=3cm]{logo_cs.jpg}
\end{center}
\end{militarybox}

\vspace{0.5cm}

% Utilisation de minipage pour créer un layout à 4 quadrants
\begin{center}
\begin{minipage}[t]{0.49\linewidth}
    % Quadrant 1 (haut gauche)
    \begin{militarybox}
    \begin{enumerate}[label=\textbf{\arabic*.}, leftmargin=1cm]
        \item Systèmes multi-robots pour simulation de champ de bataille
        \item Défis: coordination des unités, réponse aux menaces\ldots
        \item État de l'art: technologies actuelles en robotique militaire
    \end{enumerate}

    \vspace{0.5ex}
    \textbf{Objectif:} Simuler des scénarios de guerre avec flotte de robots autonomes

    \vspace{0.5ex}
    \begin{center}
    \includegraphics[width=0.9\linewidth]{example-image} % Première figure
    \end{center}
    \end{militarybox}
\end{minipage}
\hfill
\begin{minipage}[t]{0.49\linewidth}
    % Quadrant 2 (haut droit)
    \begin{militarybox}
    \begin{itemize}
        \item \textbf{Description:} Simulation d'une manœuvre d'encerclement et de neutralisation
        \item \textbf{Modèle et hypothèses:} 
        \[
            \dot{x}(t) = A x(t) + B u(t) + D p(t)
        \]
        où $p(t)$ représente les perturbations du terrain
    \end{itemize}
    
    \vspace{0.5ex}
    \begin{center}
    \includegraphics[width=0.9\linewidth]{example-image-a} % Deuxième figure
    \end{center}
    \end{militarybox}
\end{minipage}

\vspace{0.5cm}

\begin{minipage}[t]{0.49\linewidth}
    % Quadrant 3 (bas gauche)
    \begin{militarybox}
    \begin{itemize}
        \item Algorithmes de formation en V et stratégies d'évitement d'obstacles
        \item Optimisation de la trajectoire sous contraintes de sécurité et de terrain
    \end{itemize}
    
    \vspace{0.5ex}
    \begin{center}
        \footnotesize
        \rowcolors{2}{khaki!20}{darkgray!20}
        \begin{tabular}{lcc}
            \toprule
            \textbf{Critère} & \textbf{Stratégie Alpha} & \textbf{Stratégie Delta} \\
            \midrule
            Temps d'exécution & 45s & 32s \\
            Précision de tir & 87\% & 91\% \\
            Vulnérabilité & Faible & Moyenne \\
            \bottomrule
        \end{tabular}
    \end{center}
    \end{militarybox}
\end{minipage}
\hfill
\begin{minipage}[t]{0.49\linewidth}
    % Quadrant 4 (bas droit)
    \begin{militarybox}
    \begin{itemize}
        \item Taux de réussite des missions: 92\% pour les scénarios d'attaque coordonnée
        \item Analyse de la résilience face aux contre-mesures ennemies
    \end{itemize}
    
    \vspace{0.5cm}
    \subsection{Conclusions et Perspectives}
    \begin{itemize}
        \item Performances supérieures avec la stratégie d'encerclement puis neutralisation
        \item Robustesse face aux perturbations environnementales
        \item Perspectives: intégration de capacités de reconnaissance et d'apprentissage par renforcement
    \end{itemize}
    \end{militarybox}
\end{minipage}
\end{center}

\vfill
\begin{militarybox}
\begin{flushleft}
\footnotesize\textcolor{khaki}{
{[}1{]} Systèmes multi-robots dans les applications militaires (Military Robotics Journal, 2023) \hfill
{[}2{]} Algorithmes de coordination pour drones de combat \hfill
{[}3{]} Simulation de théâtres d'opérations avec flottes hétérogènes \hfill
{[}4{]} Robotarium: plateforme multi-agents pour scénarios militaires
}
\end{flushleft}
\end{militarybox}

\end{document}

\documentclass[a4paper,12pt]{article}
\usepackage[utf8]{inputenc}
\usepackage{graphicx}
\usepackage{amsmath}
\usepackage{hyperref}
\usepackage{xcolor}
\usepackage{geometry}
\usepackage{titlesec}
\usepackage{tocloft}
\usepackage{fancyhdr}

% Bibliographie
\usepackage{biblatex}
\addbibresource{references.bib}
\geometry{margin=2cm}

% Couleurs
\definecolor{militarygreen}{RGB}{53, 94, 59}
\definecolor{khaki}{RGB}{189, 183, 107}
\definecolor{darkgray}{RGB}{45, 45, 45}

% Personnalisation des sections avec couleurs militaires
\titleformat{\section}{\color{militarygreen}\Large\bfseries}{\thesection}{1em}{}
\titleformat{\subsection}{\color{khaki}\large\bfseries}{\thesubsection}{1em}{}

% Augmentation de l'écartement entre les éléments dans la table des matières
\renewcommand{\cftsecafterpnum}{\vspace{1em}} % Augmente l'espacement après les sections
\renewcommand{\cftsubsecafterpnum}{\vspace{1em}} % Augmente l'espacement après les sous-sections

% Ajout d'un pied de page propre
\pagestyle{fancy}
\fancyhf{}
\lfoot{\textbf{École CentraleSupélec}}
\rfoot{\thepage}
\renewcommand{\headrulewidth}{0pt} % Supprime la ligne de l'en-tête
\renewcommand{\footrulewidth}{0.4pt} % Ajoute une ligne au pied de page

\begin{document}

% Ajout d'une page d'entête propre avec logo
\begin{titlepage}
\begin{center}
    \includegraphics[width=0.3\textwidth]{logo_cs.jpg}\\[1cm] % Logo de l'école
    {\Huge\textbf{Systèmes Dynamiques multi-agents}}\\[0.5cm]
    \textbf{Groupe n°7\\Paul BOURGOIS, Aymeric ARNAUD, Camille GERVAIS, Joseph GUERIN}\\[1cm]
    \textbf{Date : \today}\\[1cm]
    \textbf{École CentraleSupélec}\\[1cm]
    \vfill
    \textbf{Résumé :}\\[0.5cm]
    Ce projet de robotique a exploré la manipulation de robots terrestres et de drones dans un environnement simulé, avec un objectif final d'implémentation dans une volière ROS2.\\[1cm]
    \vfill
\end{center}
\end{titlepage}

% Réduction de l'espacement pour le sommaire
\renewcommand{\cftsecfont}{\color{militarygreen}\bfseries}
\renewcommand{\cftsubsecfont}{\color{khaki}\bfseries}
\renewcommand{\cftsecafterpnum}{\vspace{-0.5em}} % Réduction de l'espacement
\renewcommand{\cftsubsecafterpnum}{\vspace{-0.5em}} % Réduction de l'espacement

% Mise à jour du sommaire avec plus de sous-parties
\tableofcontents

\newpage

\section*{Introduction}
\addcontentsline{toc}{section}{Introduction}
Ce rapport présente le projet de robotique mené sur une période de deux mois. L'objectif principal était de concevoir un scénario permettant de manipuler des robots terrestres et des drones dans un environnement simulé. Ce projet s'inscrit dans le cadre de l'étude des systèmes multi-agents et de leur coordination dans des situations complexes.

\section*{Description du Projet}
\subsection*{Applications des Grandes Flottes de Drones}

Le déploiement d'une grande flotte coordonnée de drones ouvre de toutes nouvelles possibilités dans les domaines civils et militaires. Ces systèmes apportent de l’échelle, de la rapidité et de la redondance—en particulier lorsque les moyens traditionnels sont lents, coûteux ou vulnérables. Voici les principaux domaines dans lesquels les grandes flottes de drones ont le potentiel de transformer les opérations.

\subsubsection*{Agriculture}

L’agriculture de précision a déjà adopté la technologie des drones, mais le passage à des flottes de grande taille permet de transformer des régions entières. Avec des dizaines ou des centaines de drones opérant en parallèle, les agriculteurs et agronomes peuvent :
\begin{itemize}
    \item \textbf{Cartographier et surveiller la santé des cultures} grâce à des images haute résolution en temps réel, permettant d’identifier les maladies, le stress hydrique ou les carences nutritives suffisamment tôt pour agir.
    \item \textbf{Appliquer des traitements de manière ciblée} via des pulvérisations coordonnées avec dosage variable, s’adaptant aux besoins spécifiques de chaque segment de champ—économisant le produit et réduisant le ruissellement.
    \item \textbf{Suivre le bétail et détecter des anomalies} sur de vastes pâturages ou des terrains difficiles d’accès grâce à l’imagerie thermique ou à l’analyse des schémas de mouvement.
    \item \textbf{Automatiser l’estimation des rendements} en capturant des données multispectrales au fil du temps, ce qui permet de prévoir les récoltes avec une meilleure précision.
\end{itemize}

\subsubsection*{Réponse aux Catastrophes}

Dans les heures chaotiques qui suivent une catastrophe naturelle ou humaine, une vue aérienne rapide peut sauver des vies. Une grande flotte de drones peut :
\begin{itemize}
    \item \textbf{Effectuer des recherches et sauvetages à large échelle} en couvrant des dizaines de kilomètres carrés, à l’aide de caméras infrarouges pour repérer les survivants sous les décombres ou la végétation.
    \item \textbf{Cartographier les dégâts aux infrastructures et au terrain} (ponts effondrés, zones inondées, glissements de terrain), même lorsque l’accès terrestre est bloqué ou trop dangereux.
    \item \textbf{Livrer des fournitures critiques} (médicaments, nourriture, équipements de communication) dans des zones isolées ou assiégées, avec plusieurs drones volant en parallèle ou en chaîne de relais.
    \item \textbf{Maintenir les communications} dans des zones déconnectées en agissant comme un réseau maillé aérien temporaire pour les secours ou les civils.
\end{itemize}

\subsubsection*{Logistique}

Alors que les drones sont déjà utilisés pour la livraison du dernier kilomètre, les grandes flottes offrent un modèle logistique hautement réactif et distribué. Avec l’infrastructure adéquate, elles peuvent :
\begin{itemize}
    \item \textbf{Permettre la livraison juste-à-temps} dans des environnements urbains denses ou des zones rurales, notamment là où la circulation routière ou le terrain rendent la livraison traditionnelle lente ou peu fiable.
    \item \textbf{Assurer le transport de fournitures médicales} pour des besoins urgents—comme le sang, les vaccins, les défibrillateurs—entre hôpitaux, laboratoires ou cliniques rurales.
    \item \textbf{Créer des corridors d’approvisionnement résilients} dans les régions touchées par les conflits, les pandémies ou l’isolement dû au climat.
    \item \textbf{S’adapter de manière autonome} aux pics de demande soudains (zones sinistrées, logistique d’événements) sans avoir besoin d’augmenter le personnel au sol.
\end{itemize}

\subsubsection*{Applications Militaires}

Dans un contexte militaire, une grande flotte de drones ne constitue pas seulement un outil, mais un changement de doctrine. Au lieu de dépendre de quelques plateformes coûteuses et vulnérables, des dizaines ou centaines d’unités semi-autonomes et moins onéreuses peuvent être déployées ensemble pour une redondance, une complexité et une profondeur opérationnelle accrues. Les rôles clés incluent :

\begin{itemize}
    \item \textbf{Renseignement, Surveillance \& Reconnaissance (ISR) :}  
    Les drones peuvent assurer une couverture aérienne persistante sur une zone d’intérêt, diffuser de la vidéo en direct, cartographier le terrain en 3D, et détecter des mouvements ou changements dans l’environnement, sans mettre en danger de pilotes ni dépendre des satellites.

    \item \textbf{Acquisition de cibles \& correction de tir :}  
    Équipés de désignateurs laser ou de systèmes de marquage GPS, les drones peuvent localiser des actifs ennemis et guider des munitions ou des tirs d’artillerie. Plusieurs drones observant une même cible sous différents angles permettent une triangulation plus précise et des ajustements en temps réel.

    \item \textbf{Munitions rôdeuses / Drones kamikazes :}  
    Ce sont des drones armés conçus pour survoler une zone, attendre une cible, puis frapper. Ils combinent capacité de renseignement et de frappe dans un seul dispositif—idéal contre des cibles mobiles ou des actifs de grande valeur dissimulés dans un terrain complexe.

    \item \textbf{Guerre électronique (GE) :}  
    Des drones équipés de charges utiles RF peuvent brouiller les communications ennemies, leurrer les signaux GPS pour tromper les systèmes adverses, ou agir comme des leurres mobiles pour les armes guidées par radar. En essaim, ils peuvent submerger les systèmes de détection ou générer de fausses signatures de cibles.

    \item \textbf{Relais de communication \& réseautage :}  
    En milieu dénié ou isolé, les drones peuvent former des réseaux maillés aériens pour restaurer ou étendre les communications. Cela est essentiel pour les opérations conjointes en zone sans GPS ou après brouillage. Le travail en équipe homme-machine (MUM-T) repose fortement sur ces capacités.

    \item \textbf{Combat \& Appui aérien rapproché (CAS) :}  
    Les drones armés peuvent fournir un feu de suppression, frapper des blindés ennemis ou soutenir les troupes au sol dans des zones contestées. Agissant comme des "ailiers loyaux", ils peuvent accompagner des avions pilotés, absorber les risques ou opérer de manière indépendante derrière les lignes ennemies.

    \item \textbf{Défense de base \& déni de zone :}  
    Déployés autour des bases avancées ou des infrastructures critiques, les drones peuvent patrouiller les périmètres de manière autonome, détecter les intrusions ou intercepter les menaces entrantes telles que des drones ennemis ou des véhicules—créant une défense dynamique et en couches.
\end{itemize}


\subsection*{Conception du Scénario}
\addcontentsline{toc}{subsection}{Conception du Scénario}
Le scénario développé dans ce projet implique la manipulation de robots terrestres et de drones dans un environnement simulé. L'objectif est de coordonner ces robots pour accomplir des missions complexes, telles que la formation en groupe, l'évitement d'obstacles, et la descente contrôlée des drones. Les robots sont initialement positionnés selon des configurations spécifiques pour faciliter leur coordination.

Ce scénario de contrôle des robots provient d'une mise en scène qui s'inspire des noms des robots utilisés. Le voici:\\

Il était une fois, après une discussion houleuse pendant le dîner, le général Burger et le soldat Waffle qui se sont déclarés la guerre pour savoir lequel d'entre eux vaincra Monsieur Moutarde.
Le général Burger est attendu pour une réunion très importante au Pincher pour discuter de la suite de la guerre. Cependant, ses positions ont été divulguées et le soldat Waffle va tenter de le tuer...  Le principal souci est d'avoir un moyen de transport infaillible tout en se défendant contre les délinquants.\\

Cette mise en scène, bien que fictive, montre que l'utilisation de robot terrestre et de drone prend de plus en plus d'importance dans la façon dont les grandes nations font la guerre. 
En effet, le 1\textsuperscript{er} juin 2025, l'Ukraine a mené une attaque coordonnée sur quatre bases aériennes russes en lançant 117 drones FPV, illustrant l'importance croissante des drones dans les conflits modernes. Ces systèmes, peu coûteux mais efficaces, permettent de mener des frappes ciblées à distance avec un risque humain réduit. Cette opération témoigne d'une évolution tactique vers une guerre automatisée mêlant reconnaissance, frappe et saturation ennemie~\cite{lemonde2025}.

Ce scénario utilise deux équipes de robots qui intéragissent entre eux et évoluent dans un environnement proche, il faudra donc que la gestion des collisions soient efficaces afin d'éviter tous contacts qui tout comme dans le cas d'une bataille, ferait échouer la mission.

\subsection*{Robots Utilisés et Leur Dynamique}
\addcontentsline{toc}{subsection}{Robots Utilisés et Leur Dynamique}
Les robots utilisés dans ce projet incluent :
\begin{itemize}
    \item \textbf{DCA} : Robot terrestre avec dynamique de type \textit{singleIntegrator3D}.
    \item \textbf{Burger} : Véhicule terrestre avec dynamique de type \textit{unicycle}.
    \item \textbf{Waffle} : Robot terrestre avec dynamique de type \textit{unicycle}.
    \item \textbf{RMTT} : Drone avec dynamique de type \textit{singleIntegrator3D}.
    \item \textbf{DCA\_SI} : Robot terrestre immobile ou avec dynamique de type \textit{singleIntegrator3D}.
\end{itemize}

\subsection*{Dynamiques des Robots et Défis de Contrôle}
\addcontentsline{toc}{subsection}{Dynamiques des Robots et Défis de Contrôle}
Les dynamiques des robots utilisées dans la simulation sont variées et adaptées à leurs rôles spécifiques. Voici les dynamiques principales :

\begin{itemize}
    \item \textbf{Unicycle Dynamics} :
    \begin{equation*}
    \begin{aligned}
    \dot{x} &= v \cos(\theta), \\
    \dot{y} &= v \sin(\theta), \\
    \dot{\theta} &= \omega,
    \end{aligned}
    \end{equation*}
    \noindent où $v$ est la vitesse linéaire, $\omega$ est la vitesse angulaire, et $\theta$ est l'orientation. Cette dynamique est utilisée pour les robots terrestres tels que \textit{Burger} et \textit{Waffle}. Les défis incluent :
    \begin{itemize}
        \item \textbf{Mouvements réduits} : La dynamique ne permet des mouvements de translations que dans la direction de l'avant du robot, ce qui rend certains mouvements impossibles.
        \item \textbf{Coordination} : Cette dynamique plus simple rend l'évitement de collision plus compliqué puisque certains mouvements demandent plus de temps à être effectués pour contourner.
    \end{itemize}

    \item \textbf{Single Integrator Dynamics (3D)} :
    \begin{equation*}
    \begin{aligned}
    \dot{x} &= v_x, \\
    \dot{y} &= v_y, \\
    \dot{z} &= v_z,
    \end{aligned}
    \end{equation*}
    \noindent où $v_x$, $v_y$, et $v_z$ sont les composantes de la vitesse dans les directions $x$, $y$, et $z$. Cette dynamique est utilisée pour les drones tels que \textit{RMTT} et \textit{DCA\_SI}. Les défis incluent :
    \begin{itemize}
        \item \textbf{Oscillations} : Implémenter des trajectoires oscillantes pour les drones, cette trajectoire plus compliquée demande de faire une implémentation plus approfondies en simulation.
        \item \textbf{Stabilité} : Maintenir une stabilité dans les mouvements en 3D, avec des correcteurs plus complexes afin d'optimiser le comportement.
    \end{itemize}

    \item \textbf{Stationary Dynamics} :
    \begin{equation*}
    \begin{aligned}
    \dot{x} &= 0, \\
    \dot{y} &= 0, \\
    \dot{z} &= 0,
    \end{aligned}
    \end{equation*}
    \noindent Cette dynamique est utilisée pour les robots immobiles tels que \textit{DCA\_SI} dans certaines phases de la simulation. Leur implémentation est notamment due à des contraintes de codes, en effet une flotte avec des robots de même types ne peut pas contenir qu'un seul robot. Il faut donc créer des robots immobiles pour y remédier. Les défis dans l'implémentation de ces robots incluent :
    \begin{itemize}
        \item \textbf{Interaction avec les robots mobiles} : Assurer que les robots immobiles n'interfèrent pas avec les trajectoires des robots mobiles.
    \end{itemize}
\end{itemize}

Ces dynamiques sont intégrées dans la simulation avec des lois de commande spécifiques pour répondre aux défis mentionnés. Cela permet de tester et valider les comportements des robots dans un environnement simulé avant leur déploiement réel.

\subsection*{Choix des Leaders-Followers et Stratégies de Commande}
\addcontentsline{toc}{subsection}{Choix des Leaders-Followers et Stratégies de Commande}
La stratégie \textit{Leader-Follower} a été adoptée pour coordonner les robots. Dans cette approche, un robot leader guide les autres robots (followers) vers un objectif commun. Par exemple, le robot \textit{DCA} agit comme leader, tandis que les autres robots suivent sa trajectoire en ajustant leur position et orientation.

\subsection*{Phases de la Simulation et Rôles des Robots}
\addcontentsline{toc}{subsection}{Phases de la Simulation et Rôles des Robots}
La simulation est divisée en plusieurs phases distinctes, chacune ayant des objectifs spécifiques et impliquant des rôles particuliers pour les robots :

\begin{itemize}
    \item \textbf{Phase de Formation} :
    \begin{itemize}
        \item Objectif : Les robots ajustent leur position pour maintenir une formation prédéfinie.
        \item Rôles :
        \begin{itemize}
            \item \textbf{DCA} : Agit comme leader et guide les autres robots vers la formation cible.
            \item \textbf{Burger} : Suivent le leader en ajustant leur position relative.
            \item \textbf{Waffle} : Maintient une position fixe à l'opposé de la formation pour équilibrer la structure.
            \item \textbf{RMTT} : Drone qui surveille la formation depuis une altitude élevée.
            \item \textbf{DCA\_SI} : Suiveur immobile qui reste en position pour éviter les interférences.
        \end{itemize}
    \end{itemize}

    \item \textbf{Phase Forward} :
    \begin{itemize}
        \item Objectif : Les robots avancent en ligne droite vers un objectif défini.
        \item Rôles :
        \begin{itemize}
            \item \textbf{DCA} : Continue de guider la flotte en tant que leader.
            \item \textbf{Burger} : Maintiennent leur position relative tout en avançant.
            \item \textbf{Waffle} : Suit le leader tout en restant à l'opposé de la formation.
            \item \textbf{RMTT} : Drone qui surveille les mouvements de la flotte.
            \item \textbf{DCA\_SI} : Suiveur immobile qui reste en position.
        \end{itemize}
    \end{itemize}

    \item \textbf{Phase de Danse} :
    \begin{itemize}
        \item Objectif : Les robots effectuent une danse circulaire autour du robot \textit{Waffle}.
        \item Rôles :
        \begin{itemize}
            \item \textbf{DCA} : Participe à la danse en suivant une trajectoire circulaire.
            \item \textbf{Burger} : Effectuent des mouvements synchronisés autour du \textit{Waffle}.
            \item \textbf{Waffle} : Sert de centre de la danse.
            \item \textbf{RMTT} : Drone qui surveille la danse depuis une altitude élevée.
            \item \textbf{DCA\_SI} : Reste immobile pour éviter les interférences.
        \end{itemize}
    \end{itemize}

    \item \textbf{Phase d'Oscillation des Drones} :
    \begin{itemize}
        \item Objectif : Les drones oscillent au-dessus du \textit{Waffle} pour simuler des trajectoires complexes.
        \item Rôles :
        \begin{itemize}
            \item \textbf{RMTT} : Effectue des oscillations en altitude pour surveiller la flotte.
            \item \textbf{Waffle} : Sert de point de référence pour les oscillations.
            \item \textbf{DCA}, \textbf{Burger}, \textbf{DCA\_SI} : Maintiennent leur position pour éviter les interférences.
        \end{itemize}
    \end{itemize}
\end{itemize}

Ces phases permettent de tester les capacités de coordination, de navigation et de contrôle des robots dans des scénarios variés.

\subsection*{Lois de Commande}
\addcontentsline{toc}{subsection}{Lois de Commande}
Les lois de commande utilisées dans ce projet incluent :
\begin{itemize}
    \item \textbf{Formation} : Les robots ajustent leur position pour maintenir une formation spécifique. La loi de commande est donnée par :
    \[
    \mathbf{u}_i = -k_p (\mathbf{x}_i - \mathbf{x}_\text{ref}),
    \]
    \noindent où $\mathbf{x}_i$ est la position du robot $i$, $\mathbf{x}_\text{ref}$ est la position de référence, et $k_p$ est un gain proportionnel.
    \item \textbf{Suivi de Trajectoire} : Les robots suivent une trajectoire prédéfinie en ajustant leur vitesse et orientation. La loi de commande est donnée par :
    \[
    \mathbf{u}_i = -k_p (\mathbf{x}_i - \mathbf{x}_\text{target}) + k_d \dot{\mathbf{x}}_i,
    \]
    \noindent où $\mathbf{x}_i$ est la position actuelle du robot $i$, $\mathbf{x}_\text{target}$ est la position cible, $k_p$ est un gain proportionnel, et $k_d$ est un gain dérivé pour stabiliser le mouvement.
    \item \textbf{Évitement de Collision} : Les robots détectent et évitent les obstacles en ajustant leur trajectoire. La loi de commande est donnée par :
    \[
    \mathbf{u}_i = -k_p \nabla \phi(\mathbf{x}_i),
    \]
    \noindent où $\phi(\mathbf{x}_i)$ est une fonction de potentiel qui augmente près des obstacles.
    \item \textbf{Oscillation des Drones} : Les drones oscillent au-dessus des robots terrestres en suivant des trajectoires sinusoidales. La loi de commande est donnée par :
    \[
    \begin{aligned}
    \dot{x} &= A_x \sin(\omega_x t), \\
    \dot{y} &= A_y \cos(\omega_y t), \\
    \dot{z} &= k_z (z_\text{target} - z),
    \end{aligned}
    \]
    \noindent où $A_x$ et $A_y$ sont les amplitudes, $\omega_x$ et $\omega_y$ sont les fréquences, et $k_z$ est un gain proportionnel pour la descente en $z$.
\end{itemize}

\subsection*{Zoom: Fonction de Potentiel pour l'Évitement de Collision}
\addcontentsline{toc}{subsection}{Fonction de Potentiel pour l'Évitement de Collision}
Pour assurer une navigation sûre et maintenir l'intégrité de la formation, un mécanisme d'évitement de collision est implémenté en utilisant une fonction de potentiel. Cette fonction applique des forces de répulsion aux robots suiveurs (à l'exclusion du leader et des robots stationnaires) lorsqu'ils s'approchent les uns des autres dans un rayon défini.

La force de répulsion \( u[i, :] \) pour chaque robot \( i \) est calculée comme suit :
\[
    u[i, :] += k_R \cdot \frac{(X[i, :] - X[j, :])}{\|X[i, :] - X[j, :]\|} \cdot \left(1 - \frac{\|X[i, :] - X[j, :]\|}{\text{avoidance\_radius}^2}\right)^2
\]
Où :
\begin{itemize}
    \item \( k_R \): Gain pour la force de répulsion.
    \item \( X[i, :] \): Position du robot \( i \).
    \item \( X[j, :] \): Position du robot \( j \).
    \item \( \|X[i, :] - X[j, :]\| \): Distance euclidienne entre les robots \( i \) et \( j \).
    \item \( \text{avoidance\_radius} \): Rayon dans lequel l'évitement de collision est activé.
\end{itemize}

Cette fonction de potentiel garantit que la force de répulsion augmente à mesure que la distance entre les robots diminue, empêchant ainsi les collisions. L'atténuation progressive de la force de répulsion à mesure que les robots s'éloignent est réalisée grâce au terme \( \left(1 - \frac{\|X[i, :] - X[j, :]\|}{\text{avoidance\_radius}^2}\right)^2 \)~\cite{lnr_p4}.

Ce mécanisme est crucial pour maintenir la sécurité et l'efficacité de la flotte de robots pendant les opérations.

\subsection*{Choix des coefficients des Lois de Commande}

Les lois de commandes mises en place n'ont pas été très performantes dès le début, il a donc fallu ajuster les coefficients de chaque loi de commande pour obtenir un comportement satisfaisant. La stratégie pour leur choix est la suivante:

\begin{itemize}
    \item \textbf{Formation} : Le gain proportionnel \( k_p \) a été ajusté pour assurer que les robots suivent rapidement le leader sans oscillations excessives. Un \( k_p \) trop élevé peut entraîner des oscillations, tandis qu'un \( k_p \) trop bas peut ralentir la réponse.
    \item \textbf{Suivi de Trajectoire} : Les gains proportionnels \( k_p \) et dérivés \( k_d \) ont été choisis pour équilibrer la rapidité de suivi et la stabilité. Un \( k_p \) élevé permet un suivi rapide, mais peut causer des oscillations, tandis qu'un \( k_d \) élevé aide à amortir ces oscillations. Nous avons donc augmenter la rapidité du système avec un gain proportionnel élevé et pour amortir les oscillations qui sont apparues une fois la rapidité voulues atteinte, nous avons ajouté un gain dérivé \( k_d \) pour stabiliser le mouvement.
    \item \textbf{Évitement de Collision} : Le gain de répulsion \( k_R \) a été ajusté pour assurer que les robots maintiennent une distance de sécurité sans être trop agressifs, ce qui pourrait perturber la formation. Un \( k_R \) trop élevé peut entraîner des mouvements erratiques, tandis qu'un \( k_R \) trop bas peut ne pas prévenir les collisions efficacement.
\end{itemize}

L'enjeu du choix des coefficients a été de trouver le bon équilibre entre les valeurs de coefficients entre les différentes lois de commandes. En effet, bien que la valeur absolue des coefficients soit importante, il faut que leurs valeurs relatives soient cohérentes pour que le comportement des lois de commandes combinées n'empiètent pas les uns sur les autres.

\subsection*{Implémentation de la Simulation}
\addcontentsline{toc}{subsection}{Implémentation de la Simulation}
La simulation est implémentée en utilisant une structure basée sur des classes, principalement les classes \texttt{FleetSimulation} et \texttt{Robot}. Voici une explication détaillée :

\begin{itemize}
    \item \textbf{Classe \texttt{FleetSimulation}} :
    \begin{itemize}
        \item \textbf{Objectif} : Gérer la simulation d'une flotte de robots, y compris leurs dynamiques, lois de commande, et visualisation.
        \item \textbf{Caractéristiques principales} :
        \begin{itemize}
            \item \textbf{Initialisation} : Définit la flotte de robots, le temps de début (\texttt{t0}), le temps de fin (\texttt{tf}), et le pas de temps (\texttt{dt}).
            \item \textbf{Gestion des données} : Collecte et stocke les données sur les états des robots, leurs trajectoires, et les entrées de commande au fil du temps.
            \item \textbf{Visualisation} : Fournit des méthodes pour animer les mouvements des robots en 2D et 3D, tracer les trajectoires, et afficher les entrées de commande.
        \end{itemize}
    \end{itemize}

    \item \textbf{Classe \texttt{Robot}} :
    \begin{itemize}
        \item \textbf{Objectif} : Représenter des robots individuels avec des dynamiques et des états spécifiques.
        \item \textbf{Caractéristiques principales} :
        \begin{itemize}
            \item \textbf{Initialisation} : Configure les dynamiques du robot (par exemple, \texttt{singleIntegrator2D}, \texttt{unicycle}) et son état initial.
            \item \textbf{Gestion des états} : Maintient l'état actuel du robot (par exemple, position, orientation) et le met à jour en fonction des entrées de commande.
            \item \textbf{Conversion des commandes} : Convertit les entrées de vitesse cartésienne en vitesses linéaires et angulaires pour les robots avec des dynamiques de type \texttt{unicycle}.
        \end{itemize}
    \end{itemize}

    \item \textbf{Workflow de la Simulation} :
    \begin{itemize}
        \item \textbf{Création de Flottes} : Les robots sont instanciés à l'aide de la classe \texttt{Robot} et regroupés en flottes. Chaque flotte est assignée des dynamiques spécifiques et des positions initiales.
        \item \textbf{Algorithmes de Commande} : Les lois de commande (par exemple, formation, suivi de trajectoire, évitement de collision) sont appliquées pour calculer les entrées de vitesse pour chaque robot. Des algorithmes spécialisés comme \texttt{drone\_oscillation\_3d} gèrent des comportements complexes tels que les oscillations des drones.
        \item \textbf{Intégration} : La classe \texttt{FleetSimulation} intègre le mouvement de tous les robots au fil du temps en utilisant les entrées de commande calculées.
        \item \textbf{Visualisation} : La simulation fournit des animations en temps réel des mouvements des robots en 2D et 3D. Les trajectoires, états, et entrées de commande sont tracés pour l'analyse.
    \end{itemize}
\end{itemize}

Cette structure permet de tester les capacités de coordination, de navigation, et de contrôle des robots dans des scénarios variés.

\subsection*{Implémentation dans ROS2}
\addcontentsline{toc}{subsection}{Implémentation dans ROS2}
Une fois la simulation fonctionnelle, il a fallu comprendre la fonctionnement de l'implémentation en ROS2 dans la volière. Nous avons tout de suite identifié les attendus et les rendus de la boucle de la machine à  état qui dicte le fonctionnement des robots. Notre code fonctionne ainsi:

\begin{itemize}
    \item \textbf{En entrée} : La simulation reçoit la position absolue de tous les robots dans le repère de la volière. Cette position est donnée par une odométrie absolue qui fonctionne avec 16 caméras.
    \item \textbf{En sortie} : La simulation envoie les vitesses attendues pour les robots dans le repère de la volière. Ces vitesses sont calculées en fonction des dynamiques des robots et des lois de commande appliquées.
\end{itemize}

Cette implémentation est censée être légère en code puisque la simulation a été construite en connaissant le fonctionnement de la colière. Ainsi, il faut juste entrer les fonctions de commande dans le fonctionnement de la volière et sans les rendus de simulation pour faire fonctionner la simulation.\\

Cette simplicité n'est bien sûr pas si évidente et lors de la mise en place dans la simulation nous avons du faire face à plusieurs challenges:

\begin{itemize}
    \item \textbf{Grand nombre de robots} : Notre code et notre scénario utilise un grand nomnbre de robots et pour simplifier la mise en place du code, il nous a été demandé de ne mettre que certains robots. Nous avons alors fait face aux limitations de notre code en remarquant que celui-ci ne s'adaptait pas bien aux changements de nombre de robots. 
    \item \textbf{Repères} : Il a fallu s'assurer que les repères utilisés correponsdaient bien entre la simulation et la volière. En effet si les repères sont différents, les écarts de position seront énormes puisque l'odométrie est absolue et les réponses de commande seront aberrantes.
\end{itemize}

Globalement les essais que nous avons fait dans la volière nous montre la dure réalité de la mise en place de simulations dans un système réel, bien que contrôlé dans le cadre de la volière. La gestion de l'écosystème ROS2 demande de la rigueur dans les codes qui sont utilisés et une bonne compréhension du système global qui gère les commandes de robot.


\section*{Conclusion}
\addcontentsline{toc}{section}{Conclusion}
Ce projet de robotique a permis d'explorer la coordination de robots terrestres et de drones dans un environnement simulé. Les principales réalisations incluent :

\begin{itemize}
    \item L'implémentation réussie de différentes dynamiques de robots (unicycle, single integrator)
    \item La mise en place de stratégies de formation et d'évitement de collisions
    \item Le développement de lois de commande robustes pour la coordination des robots
    \item La compréhension des défis liés à l'implémentation dans un environnement ROS2, notamment dans la volière
\end{itemize}

Ce projet permet d'appréhender les défis de la commande de plusieurs robots ensemble. Notre choix s'est porté sur le développement d'un scénario avec un nombre importants de robots et des interactions complexes (changement de formations, changement de leader) et non sur une intelligence dans les déplacements. Ce choix nous a forcé à construire des lois de commande complexes pour appréhender les différents cas extrêmes liés à notre scénario. Ce choix nous a aussi demandé de comprendre les mécanismes et les interactions entre les différents termes de la loi de commande pour comprendre les comportements des robots et les adapter.

% Ajout de la section Bibliographie
\newpage
\section*{Bibliographie}
\addcontentsline{toc}{section}{Bibliographie}
\printbibliography[heading=none]

\end{document}